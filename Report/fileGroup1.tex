\subsection{The next evolution of MDE: a seamless integration of machine
learning into domain modeling}
Machine learning (ML) is a widely used approach for enabling computers to learn from and make decisions based on data. There are many tools and frameworks available for implementing ML algorithms, such as TensorFlow, GraphLab, and Infer.NET. These tools allow for the expression of ML algorithms at a higher level of abstraction, and often include an execution engine for running the algorithms on a variety of devices.

There is also a growing trend towards using a model-based approach for ML, in which ML problems are specified using a dedicated modeling language and the corresponding ML code is generated automatically. This allows for the creation of highly tailored models for specific scenarios and rapid prototyping and comparison of alternative models.

Incremental learning methods, such as those based on Hoeffding bounds and Hoeffding trees, have been proposed as a way to efficiently process and analyze massive data streams in real-time. Tools like MOA provide implementation and support for these methods.

Other research has focused on weaving ML into domain modeling, allowing ML algorithms to be seamlessly integrated into the process of creating models for a particular domain or application. This approach involves decomposing ML into small, reusable units called microlearning units, which can be modeled and used alongside domain data. This approach allows for the flexible combination of learned behaviors and domain knowledge, and can be more accurate and efficient than learning global behaviors.

In this paper \cite {}, the authors propose a method for seamlessly integrating machine learning into domain modeling by decomposing machine learning into microlearning units that are modeled together with and at the same level as the domain data. The approach is demonstrated using a smart grid case study, showing that it can be significantly more accurate than learning a global behavior, while still being fast enough for live learning.

\subsection{ModelSet: a dataset for machine learning in model-driven engineering}
The application of machine learning (ML) to software engineering has gained significant attention in recent years, with a focus on three types of ML models: code-generating models, representational models of code, and pattern mining models. These models have various applications, such as code recommendation systems, inferring coding conventions, clone detection, and code-to-text and text-to-code translation. The use of ML in model-driven engineering (MDE) has not been fully explored yet, and this paper \cite{} aims to encourage researchers to adapt existing ML models for handling modeling artifacts using the ModelSet dataset.

There are several existing datasets related to software models, including the LindholmenDataset and ModelSet, as well as datasets of OCL expressions, BPMN models, and APIs classified using Maven Central tags. However, many of these datasets have limitations, such as a variety of formats and versions, invalid or poor quality models, or a lack of labels.

There have been several studies on the application of AI and ML to address MDE problems, including search-based algorithms and reinforcement learning for co-evolution and model repair, respectively. There have also been efforts to classify UML class diagrams and Ecore meta-models, and to use clustering techniques on collections of models. However, many of these studies are not easily replicable due to the lack of available datasets or the difficulty in processing the models.

This paper \cite{} presents the ModelSet dataset, which includes over 10,000 labeled models and aims to address the need for a large and high-quality dataset of software models for use in ML research. The dataset is composed of models in the Ecore metamodeling language, and includes labels for the model category and domain, as well as structural and textual features. The authors also demonstrate the use of the dataset in a classification task, showing that it can be used to improve the accuracy of existing ML models for classifying Ecore models.
